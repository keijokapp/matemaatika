\documentclass[fleqn]{article}

\usepackage{tikz}
\usepackage{amsmath}
\usepackage{pgfplots}
\usepgfplotslibrary{fillbetween}
\usetikzlibrary{patterns}
\usetikzlibrary{shapes,shadows,arrows,positioning,graphs}

\pgfplotsset{compat=1.12}

\setlength{\mathindent}{1cm}

\newcommand*\ex[1]{%
	\clearpage
	\begin{tikzpicture}
		\node[draw,circle,inner sep=2pt, thick] {#1};
	\end{tikzpicture}}


\begin{document}

\ex{1.}
\begin{displaymath}
|x - 1| + x > |2x + 1|
\end{displaymath}
\begin{displaymath}
|x - 1| + x - |2x + 1| > 0 
\end{displaymath}
\begin{tikzpicture}
	\begin{axis}[axis lines=middle,axis on top, grid=both,no markers,xlabel=$x$,ylabel=$y$,
		xmin=-5,xmax=5,
		ymin=-5,ymax=5,
		xtick={-5,...,5},
		ytick={-5,...,5},
        ]

		\addplot+[thick, samples=10,name path=function1] {(x - 1) + x - (2*x + 1)};
		\addplot+[thick, samples=10,name path=function2] {-(x - 1) + x - (2*x + 1)};
		\addplot+[thick, samples=10,name path=function3] {(x - 1) + x - -(2*x + 1)};
		\addplot+[thick, samples=10,name path=function4] {-(x - 1) + x - -(2*x + 1)};
		\addplot+[name path=axis] coordinates {(-10,0) (10,0)};
		\addplot[white] fill between[ of=function4 and axis,split,
                every segment no 1/.style={green!20}];
		\addplot[transparent] fill between[ of=function3 and axis,split,
                every segment no 1/.style={opacity=1,white}];
		\addplot[transparent] fill between[ of=function4 and function2,split,
                every segment no 1/.style={opacity=1,white}];
    \end{axis}
\end{tikzpicture}
\begin{displaymath}
x \in (-1, 0)
\end{displaymath}

\ex{2.}
\begin{displaymath}
|x| < |x + 2|
\end{displaymath}
\begin{displaymath}
|x| - |x + 2| < 0
\end{displaymath}
\begin{tikzpicture}
	\begin{axis}[axis lines=middle,axis on top, grid=both,no markers,xlabel=$x$,ylabel=$y$,
		xmin=-5,xmax=5,
		ymin=-5,ymax=5,
		xtick={-5,...,5},
		ytick={-5,...,5},
        ]

		\addplot+[thick, samples=10,name path=function1] {(x) - (x + 2)};
		\addplot+[thick, samples=10,name path=function2] {-(x) - (x + 2)};
		\addplot+[thick, samples=10,name path=function3] {(x) - -(x + 2)};
		\addplot+[thick, samples=10,name path=function4] {-(x) - -(x + 2)};
		\addplot+[name path=axis] coordinates {(-10,0) (10,0)};
		\addplot[green!20] fill between[ of=function2 and axis,split,
                every segment no 0/.style={opacity=1,white}];
		\addplot[white] fill between[ of=function1 and function2];
    \end{axis}
\end{tikzpicture}
\begin{displaymath}
x \in (-1, \infty)
\end{displaymath}

\ex{2.}
\begin{displaymath}
|x + 2| - |x - 2| \leq x - 1
\end{displaymath}
\begin{displaymath}
|x + 2| - |x - 2| - x + 1 \leq 0
\end{displaymath}
\begin{tikzpicture}
	\begin{axis}[axis lines=middle,axis on top, grid=both,no markers,xlabel=$x$,ylabel=$y$,
		xmin=-7,xmax=7,
		ymin=-7,ymax=7,
		xtick={-7,...,7},
		ytick={-7,...,7},
		domain=-10:10
        ]

		\addplot+[thick, samples=10,name path=function1] {(x+2) - (x-2) - x + 1};
		\addplot+[thick, samples=10,name path=function2] {-(x+2) - (x-2) - x + 1};
		\addplot+[thick, samples=10,name path=function3] {(x+2) - -(x-2) - x + 1};
		\addplot+[thick, samples=10,name path=function4] {-(x+2) - -(x-2) - x + 1};
		\addplot+[name path=axis] coordinates {(-10,0) (10,0)};
		\addplot[green!20] fill between[ of=function4 and axis,split,
		         every segment no 0/.style={opacity=1,white}];
		\addplot[white] fill between[ of=function3 and function4];
		\addplot[green!20] fill between[ of=function1 and axis,split,
		         every segment no 0/.style={opacity=1,white}];
    \end{axis}
\end{tikzpicture}
\begin{displaymath}
x \in [-3, -1] \cup [5; \infty)
\end{displaymath}

\ex{4.}
\begin{displaymath}
2|x + 1| > 3x - |x + 2|
\end{displaymath}
\begin{displaymath}
2|x + 1| - 3x + |x + 2| > 0
\end{displaymath}
\begin{tikzpicture}
	\begin{axis}[axis lines=middle,axis on top, grid=both,no markers,xlabel=$x$,ylabel=$y$,
		xmin=-7,xmax=7,
		ymin=-7,ymax=7,
		xtick={-7,...,7},
		ytick={-7,...,7},
		domain=-10:10
        ]

		\addplot+[thick, samples=10,name path=function1] {2*(x+1) - 3*x + (x+2)};
		\addplot+[thick, samples=10,name path=function2] {2*-(x+1) - 3*x + (x+2)};
		\addplot+[thick, samples=10,name path=function3] {2*(x+1) - 3*x + -(x+2)};
		\addplot+[thick, samples=10,name path=function4] {2*-(x+1) - 3*x + -(x+2)};
		\addplot+[name path=axis] coordinates {(-10,0) (10,0)};
		\addplot[green!20] fill between[ of=function1 and axis];
    \end{axis}
\end{tikzpicture}
\begin{displaymath}
x \in (-\infty; \infty)
\end{displaymath}

\ex{5.}
\begin{displaymath}
|x| > x
\end{displaymath}
\begin{displaymath}
|x| - x > 0
\end{displaymath}
\begin{tikzpicture}
	\begin{axis}[axis lines=middle,axis on top, grid=both,no markers,xlabel=$x$,ylabel=$y$,
		xmin=-7,xmax=7,
		ymin=-7,ymax=7,
		xtick={-7,...,7},
		ytick={-7,...,7},
		domain=-10:10
        ]

		\addplot+[thick, samples=10,name path=function1] {x - x};
		\addplot+[thick, samples=10,name path=function2] {-x - x};
		\addplot[green!20] fill between[ of=function1 and function2,split,
		         every segment no 1/.style={opacity=1,white}];
    \end{axis}
\end{tikzpicture}
\begin{displaymath}
x \in (-\infty; 0)
\end{displaymath}

\ex{6.}
\begin{displaymath}
|2x - 1| < |1 - x|
\end{displaymath}
\begin{displaymath}
|2x - 1| - |1 - x| < 0
\end{displaymath}
\begin{tikzpicture}
	\begin{axis}[axis lines=middle,axis on top, grid=both,no markers,xlabel=$x$,ylabel=$y$,
		xmin=-2,xmax=2,
		ymin=-2,ymax=2,
		xtick={-2,...,2},
		ytick={-2,...,2},
		domain=-5:5
        ]

		\addplot+[thick, samples=10,name path=function1] {(2*x-1) - (1-x)};
		\addplot+[thick, samples=10,name path=function2] {-(2*x-1) - (1-x)};
		\addplot+[thick, samples=10,name path=function3] {(2*x-1) - -(1-x)};
		\addplot+[thick, samples=10,name path=function4] {-(2*x-1) - -(1-x)};
		\addplot+[name path=axis] coordinates {(-5,0) (5,0)};
		\addplot[green!20] fill between[ of=function2 and axis,split,
		         every segment no 0/.style={opacity=1,white}];
		\addplot[transparent] fill between[ of=function1 and function2,split,
		         every segment no 1/.style={opacity=1,white}];
    \end{axis}
\end{tikzpicture}
\begin{displaymath}
x \in (0; \frac{2}{3})
\end{displaymath}

\ex{7.}
\begin{displaymath}
|x - 3| - |2 - x| \geq |x - 1|
\end{displaymath}
\begin{displaymath}
|x - 3| - |2 - x| - |x - 1| \geq 0
\end{displaymath}
\begin{tikzpicture}
	\begin{axis}[axis lines=middle,axis on top, grid=both,no markers,xlabel=$x$,ylabel=$y$,
		xmin=-2,xmax=6,
		ymin=-4,ymax=4,
		xtick={-2,...,6},
		ytick={-4,...,4},
		domain=-2:6
        ]

		\addplot+[thick, samples=10,name path=function1] {+(x - 3) - +(2 - x) - +(x - 1)};
		\addplot+[thick, samples=10,name path=function2] {-(x - 3) - +(2 - x) - +(x - 1)};
		\addplot+[thick, samples=10,name path=function3] {+(x - 3) - -(2 - x) - +(x - 1)};
		\addplot+[thick, samples=10,name path=function4] {-(x - 3) - -(2 - x) - +(x - 1)};
		\addplot+[thick, samples=10,name path=function5] {+(x - 3) - +(2 - x) - -(x - 1)};
		\addplot+[thick, samples=10,name path=function6] {-(x - 3) - +(2 - x) - -(x - 1)};
		\addplot+[thick, samples=10,name path=function7] {+(x - 3) - -(2 - x) - -(x - 1)};
		\addplot+[thick, samples=10,name path=function8] {-(x - 3) - -(2 - x) - -(x - 1)};
		\addplot+[name path=axis] coordinates {(-5,0) (5,0)};
		\addplot[green!20] fill between[ of=function2 and axis,split,
		         every segment no 1/.style={opacity=1,white}];
		\addplot[transparent] fill between[ of=function2 and function6,split,
		         every segment no 0/.style={opacity=1,white}];
    \end{axis}
\end{tikzpicture}
\begin{displaymath}
x \in [0; 2]
\end{displaymath}

\ex{8.}
\begin{displaymath}
|\frac{1 - x}{x + 1}| \geq 1
\end{displaymath}
\begin{displaymath}
|\frac{1 - x}{x + 1}| - 1 \geq 0
\end{displaymath}
\begin{tikzpicture}
	\begin{axis}[axis lines=middle,axis on top, grid=both,no markers,xlabel=$x$,ylabel=$y$,
		xmin=-4,xmax=4,
		ymin=-4,ymax=4,
		xtick={-4,...,4},
		ytick={-4,...,4},
		domain=-4:4
        ]

		\addplot+[thick, samples=100,name path=function1] {+(1-x)/(x+1) - 1};	
		\addplot+[thick, samples=100,name path=function2] {-(1-x)/(x+1) - 1};
		\draw (-1,0) circle (1mm);
		\addplot+[name path=axis] coordinates {(-5,0) (5,0)};
		\addplot[green!20] fill between[ of=function1 and axis,split,
		         every segment no 0/.style={opacity=1,white},
		         every segment no 2/.style={opacity=1,white}];
		\addplot[green!20] fill between[ of=function2 and axis,split,
		         every segment no 1/.style={opacity=1,white},
		         ];
    \end{axis}
\end{tikzpicture}
\begin{displaymath}
x \in (-\infty;-1) \cup (-1; 0]
\end{displaymath}

\ex{9.}
\begin{displaymath}
|\frac{2x - 1}{x - 1}| \geq 2
\end{displaymath}
\begin{displaymath}
|\frac{2x - 1}{x - 1}| - 2 \geq 0
\end{displaymath}
\begin{tikzpicture}
	\begin{axis}[axis lines=middle,axis on top, grid=both,no markers,xlabel=$x$,ylabel=$y$,
		xmin=-4,xmax=7,
		ymin=-7,ymax=4,
		xtick={-4,...,7},
		ytick={-7,...,4},
		domain=-10:10
        ]

		\addplot+[thick, samples=100,name path=function1] {+(2*x-1)/(x-1) - 2};	
		\addplot+[thick, samples=100,name path=function2] {-(2*x-1)/(x-1) - 2};	
		\draw (1,0) circle (1mm);
		\addplot+[name path=axis] coordinates {(-5,0) (5,0)};
		\addplot[green!20] fill between[ of=function1 and axis,split,
		         every segment no 0/.style={opacity=1,white}];
		\addplot[green!20] fill between[ of=function2 and axis,split,
		         every segment no 0/.style={opacity=1,white},
		         every segment no 2/.style={opacity=1,white}];
    \end{axis}
\end{tikzpicture}
\begin{displaymath}
x \in [\frac{3}{4};1) \cup (1; \infty)
\end{displaymath}

\ex{10.}
\begin{displaymath}
\frac{1}{|x - 2|} < \frac{2}{|x + 1|}
\end{displaymath}
\begin{displaymath}
\frac{1}{|x - 2|} - \frac{2}{|x + 1|} < 0
\end{displaymath}
\begin{tikzpicture}
	\begin{axis}[axis lines=middle, grid=both,no markers,xlabel=$x$,ylabel=$y$,
		xmin=-4,xmax=6,
		ymin=-6,ymax=4,
		xtick={-4,...,6},
		ytick={-6,...,4},
		domain=-10:10
        ]

		\addplot+[thick, samples=500,name path=function1] {1/+(x-2) - 2/+(x+1)};
		\addplot+[thick, samples=500,name path=function2] {1/-(x-2) - 2/+(x+1)};
		\addplot+[thick, samples=500,name path=function3] {1/+(x-2) - 2/-(x+1)};
		\addplot+[thick, samples=500,name path=function4] {1/-(x-2) - 2/-(x+1)};
		\draw (-1,0) circle (1mm);
		\draw (2,0) circle (1mm);
		\node[white, fill=black, text opacity=1, fill opacity=.2, text width=4cm, align=center] at (1,-1) {\Large\bf\fontfamily{qcr}\selectfont I have no idea what I'm doing};
    \end{axis}
\end{tikzpicture}
\begin{displaymath}
x \in (-\infty;-1) \cup (-1; 1) \cup (5; \infty)
\end{displaymath}

\ex{11.}
\begin{displaymath}
\frac{|x| - 1}{|x - 1|} \geq 1
\end{displaymath}
\begin{displaymath}
\frac{|x| - 1}{|x - 1|} - 1 \geq 0
\end{displaymath}
\begin{tikzpicture}
	\begin{axis}[axis lines=middle,axis on top, grid=both,no markers,xlabel=$x$,ylabel=$y$,
		xmin=-4,xmax=5,
		ymin=-5,ymax=4,
		xtick={-4,...,5},
		ytick={-5,...,4},
		domain=-10:10
        ]

		\addplot+[thick, samples=100,name path=function1] {(+x - 1) / +(x - 1) - 1};
		\addplot+[thick, samples=100,name path=function2] {(-x - 1) / +(x - 1) - 1};
		\addplot+[thick, samples=100,name path=function3] {(+x - 1) / -(x - 1) - 1};
		\addplot+[thick, samples=100,name path=function4] {(-x - 1) / -(x - 1) - 1};
		\draw (1,0) circle (1mm);
		\addplot[green!20] fill between[ of=function1 and function4,split,
		         every segment no 0/.style={opacity=1,white}];
    \end{axis}
\end{tikzpicture}
\begin{displaymath}
x \in (1, \infty)
\end{displaymath}

\ex{12.}
\begin{displaymath}
x^2 + x - 2 < 0
\end{displaymath}
\begin{displaymath}
x \in \{-2; 1\}
\end{displaymath}

\begin{tikzpicture}
	\begin{axis}[axis lines=middle,axis on top, grid=both,no markers,xlabel=$x$,ylabel=$y$,
		xmin=-5,xmax=5,
		ymin=-5,ymax=5,
		xtick={-5,...,5},
		ytick={-5,...,5},
        ]

		\addplot+[thick, samples=100,name path=function] {x^2+x-2};

		\addplot+[name path=axis] coordinates {(-10,0) (10,0)};

		\addplot[white] fill between[ of=function and axis,split,
                every segment no 1/.style={green!20}];
    \end{axis}
\end{tikzpicture}
\begin{displaymath}
x \in (-2, 1)
\end{displaymath}

\ex{13.}
\begin{displaymath}
x^2 - 3x + 2 \geq 0
\end{displaymath}
\begin{displaymath}
x \in \{1; 2\}
\end{displaymath}

\begin{tikzpicture}
	\begin{axis}[axis lines=middle,axis on top, grid=both,no markers,xlabel=$x$,ylabel=$y$,
		xmin=-5,xmax=5,
		ymin=-5,ymax=5,
		xtick={-5,...,5},
		ytick={-5,...,5},
        ]

		\addplot+[thick, samples=100,name path=function] {x^2-3*x+2};

		\addplot+[name path=axis] coordinates {(-10,0) (10,0)};

		\addplot[white] fill between[ of=function and axis,split,
                every segment no 0/.style={green!20},
                every segment no 2/.style={green!20}];
    \end{axis}
\end{tikzpicture}
\begin{displaymath}
x \in (-\infty, 1] \cup [2; \infty) 
\end{displaymath}

\ex{14.}
\begin{displaymath}
x^2 + 2x + 2 > 0
\end{displaymath}
\begin{displaymath}
x \in \emptyset
\end{displaymath}

\begin{tikzpicture}
	\begin{axis}[axis lines=middle,axis on top, grid=both,no markers,xlabel=$x$,ylabel=$y$,
		xmin=-5,xmax=5,
		ymin=-5,ymax=5,
		xtick={-5,...,5},
		ytick={-5,...,5},
        ]

		\addplot+[thick, samples=100,name path=function] {x^2+2*x+2};

		\addplot+[name path=axis] coordinates {(-10,0) (10,0)};

		\addplot[green!20] fill between[ of=function and axis ];
    \end{axis}
\end{tikzpicture}
\begin{displaymath}
x \in (-\infty, \infty)
\end{displaymath}

\ex{15}
\begin{displaymath}
-x^2 + 4x - 4 > 0
\end{displaymath}
\begin{displaymath}
x_1 = x_2 = 2
\end{displaymath}
\begin{tikzpicture}
	\begin{axis}[axis lines=middle,axis on top, grid=both,no markers,xlabel=$x$,ylabel=$y$,
		xmin=-5,xmax=5,
		ymin=-5,ymax=5,
		xtick={-5,...,5},
		ytick={-5,...,5},
        ]
		\addplot+[thick, samples=100,name path=function] {-x^2+4*x-4};
		\addplot+[name path=axis] coordinates {(-10,0) (10,0)};
    \end{axis}
\end{tikzpicture}
\begin{displaymath}
x \in \emptyset
\end{displaymath}

\ex{16}
\begin{displaymath}
-x^2 + 4x - 4 \leq 0
\end{displaymath}
\begin{displaymath}
x_1 = x_2 = 2
\end{displaymath}
\begin{tikzpicture}
	\begin{axis}[axis lines=middle,axis on top, grid=both,no markers,xlabel=$x$,ylabel=$y$,
		xmin=-5,xmax=5,
		ymin=-5,ymax=5,
		xtick={-5,...,5},
		ytick={-5,...,5},
        ]
		\addplot+[thick, samples=100,name path=function] {-x^2+4*x-4};
		\addplot+[name path=axis] coordinates {(-10,0) (10,0)};
		\addplot[green!20] fill between[ of=function and axis ];
    \end{axis}
\end{tikzpicture}
\begin{displaymath}
x \in (\infty, -\infty)
\end{displaymath}

\ex{17}
\begin{displaymath}
x^2-|x|-6 < 0
\end{displaymath}
\begin{tikzpicture}
	\begin{axis}[axis lines=middle,axis on top, grid=both,no markers,xlabel=$x$,ylabel=$y$,
		xmin=-5,xmax=5,
		ymin=-9,ymax=9,
		xtick={-5,...,5},
		ytick={-9,...,9},
        ]
		\addplot+[thick, samples=100,name path=function] {x^2 - x - 6};
		\addplot+[thick, samples=100,name path=function2] {x^2 - -x - 6};
		\addplot+[name path=axis] coordinates {(-10,0) (10,0)};
		\addplot[white] fill between[ of=function and axis,split,
                every segment no 1/.style={green!20}];
		\addplot[white] fill between[ of=function2 and axis,split,
                every segment no 1/.style={green!20}];
    \end{axis}
\end{tikzpicture}
\begin{displaymath}
x \in (-3; 3)
\end{displaymath}

\ex{18}
\begin{displaymath}
x^2 - 2|x + 2| - 4 \leq 0
\end{displaymath}
\begin{tikzpicture}
	\begin{axis}[axis lines=middle,axis on top, grid=both,no markers,xlabel=$x$,ylabel=$y$,
		xmin=-5,xmax=5,
		ymin=-9,ymax=9,
		xtick={-5,...,5},
		ytick={-9,...,9},
        ]
		\addplot+[thick, samples=100,name path=function] {x^2- 2*(x+2) - 4};
		\addplot+[thick, samples=100,name path=function2] {x^2- 2*-(x+2) - 4};
		\addplot+[name path=axis] coordinates {(-10,0) (10,0)};
		\addplot[white] fill between[ of=function and axis,split,
                every segment no 1/.style={green!20}];
		\addplot[white] fill between[ of=function2 and axis,split,
                every segment no 1/.style={green!20}];
    \end{axis}
\end{tikzpicture}
\begin{displaymath}
x \in [-2; 4]
\end{displaymath}

\ex{19}
\begin{displaymath}
x^2 - |4x - 5| > x - 1
\end{displaymath}
\begin{displaymath}
x^2 - |4x - 5| - x + 1 > 0
\end{displaymath}
\begin{tikzpicture}
	\begin{axis}[axis lines=middle,axis on top, grid=both,no markers,xlabel=$x$,ylabel=$y$,
		xmin=-5,xmax=5,
		ymin=-9,ymax=9,
		xtick={-5,...,5},
		ytick={-9,...,9},
        ]
		\addplot+[thick, samples=100,name path=function] {x^2 - (4*x-5) - x + 1};
		\addplot+[thick, samples=100,name path=function2] {x^2 - -(4*x-5) - x + 1};
		\addplot+[name path=axis] coordinates {(-10,0) (10,0)};
		\addplot[white] fill between[ of=function and axis,split,
                every segment no 0/.style={green!20},
                every segment no 2/.style={green!20}];
		\addplot[white] fill between[ of=function2 and axis,split,
                every segment no 0/.style={green!20},
                every segment no 2/.style={green!20}];
		\addplot[white] fill between[ of=function and function2 ];
    \end{axis}
\end{tikzpicture}
\begin{displaymath}
x \in (-\infty; -4)\cup(1; 2)\cup(3; \infty)
\end{displaymath}

\ex{20}
\begin{displaymath}
|5x + 3| > x^2 + 2x + 3
\end{displaymath}
\begin{displaymath}
|5x + 3| - x^2 - 2x - 3 > 0
\end{displaymath}
\begin{tikzpicture}
	\begin{axis}[axis lines=middle,axis on top, grid=both,no markers,xlabel=$x$,ylabel=$y$,
		xmin=-7,xmax=7,
		ymin=-7,ymax=7,
		xtick={-7,...,7},
		ytick={-7,...,7},
        ]
		\addplot+[thick, domain=-8:8, samples=100,name path=function] {(5*x + 3) - x^2 - 2*x - 3};
		\addplot+[thick, domain=-8:8, samples=100,name path=function2] {-(5*x + 3) - x^2 - 2*x - 3};
		\addplot+[name path=axis] coordinates {(-10,0) (10,0)};
		\addplot[white] fill between[ of=function and axis,split,
                every segment no 1/.style={green!20}];
		\addplot[white] fill between[ of=function2 and axis,split,
                every segment no 1/.style={green!20}];
    \end{axis}
\end{tikzpicture}
\begin{displaymath}
x \in (-6; -1)\cup(0; 3)
\end{displaymath}

\end{document}