\documentclass[fleqn]{article}

\usepackage{tikz}
\usepackage{amsmath}
\usepackage{pgfplots}
\usepgfplotslibrary{fillbetween}
\usetikzlibrary{patterns}
\usetikzlibrary{shapes,shadows,arrows,positioning,graphs}

\pgfplotsset{compat=1.12}

\setlength{\mathindent}{1cm}

\newcommand*\ex[1]{%
	\clearpage
	\begin{tikzpicture}
		\node[draw,circle,inner sep=2pt, thick]{#1};
	\end{tikzpicture}}


\begin{document}

\ex{18.}
\begin{displaymath}
y = x^2; x \in [0; \infty)
\end{displaymath}
\begin{displaymath}
x = \sqrt{y};
\end{displaymath}
\begin{tikzpicture}
	\begin{axis}[axis lines=middle,grid=both,no markers,xlabel=$x$,ylabel=$y$,
		xmin=-1,xmax=5,
		ymin=-1,ymax=5,
		xtick={-1,...,5},
		ytick={-1,...,5},
        ]

		\draw[black] (-2, -2) -- (10, 10);
		\addplot+[thick, domain=0:10, samples=100] {x^2};
		\addplot+[thick, domain=0:10, samples=100] {sqrt(x)};
    \end{axis}
\end{tikzpicture}
\begin{displaymath}
Y = [0, \infty); X = [0, \infty);
\end{displaymath}

\ex{19.}
\begin{displaymath}
y = x^2 - 2x - 3; x \in [1; \infty)
\end{displaymath}
\begin{displaymath}
x^2 - 2x - 3 - y = 0
\end{displaymath}
\begin{displaymath}
x^2 - 2x - (3 + y) = 0
\end{displaymath}
\begin{displaymath}
x = \frac{2 + \sqrt{4 + 4(3 + y)}}{2}
\end{displaymath}
\begin{displaymath}
x = \frac{2 + \sqrt{4(4 + y)}}{2}
\end{displaymath}
\begin{displaymath}
x = \frac{2 + 2\sqrt{4 + y}}{2}
\end{displaymath}
\begin{displaymath}
x = 1 + \sqrt{4 + y}; y \in (-4; \infty]
\end{displaymath}
\begin{tikzpicture}
	\begin{axis}[axis lines=middle,grid=both,no markers,xlabel=$x$,ylabel=$y$,
		xmin=-5,xmax=7,
		ymin=-5,ymax=7,
		xtick={-5,...,7},
		ytick={-5,...,7},
        ]

		\draw[black] (-10, -10) -- (10, 10);
		\addplot+[thick, domain=1:10, samples=100] {x^2 - 2*x - 3};
		\addplot+[thick, domain=-4:10, samples=100] {1 + sqrt(4 + x)};
    \end{axis}
\end{tikzpicture}
\begin{displaymath}
Y = [-4, \infty); X = [1, \infty)
\end{displaymath}

\ex{20.}
\begin{displaymath}
y = x^2 - 3x - 4; x \in (-\infty; 1,5]
\end{displaymath}
\begin{displaymath}
x^2 - 3x - 4 - y = 0
\end{displaymath}
\begin{displaymath}
x^2 - 3x - (4 + y) = 0
\end{displaymath}
\begin{displaymath}
x = \frac{3 - \sqrt{9 + 4(4 + y)}}{2}
\end{displaymath}
\begin{displaymath}
x = \frac{3 - \sqrt{25 + 4y}}{2}
\end{displaymath}
\begin{displaymath}
x = 1,5 - \sqrt{\frac{25 + 4y}{4}}
\end{displaymath}
\begin{displaymath}
x = 1,5 - \sqrt{6,25 + y}; y \in [-6.25; \infty)
\end{displaymath}
\begin{tikzpicture}
	\begin{axis}[axis lines=middle,grid=both,no markers,xlabel=$x$,ylabel=$y$,
		xmin=-7,xmax=6,
		ymin=-7,ymax=6,
		xtick={-10,...,10},
		ytick={-10,...,10},
        ]

		\draw[black] (-10, -10) -- (10, 10);
		\addplot+[thick, domain=-10:1.5, samples=100] {x^2 - 3*x - 4};
		\addplot+[thick, domain=-6.25:10, samples=100] {1.5 - sqrt(6.25 + x)};
    \end{axis}
\end{tikzpicture}
\begin{displaymath}
Y = [-6,25; \infty); X = (-\infty; 1,5]
\end{displaymath}

\ex{21.}
\begin{displaymath}
y = x^3 + 2
\end{displaymath}
\begin{displaymath}
x^3 = y - 2
\end{displaymath}
\begin{displaymath}
x = \sqrt[3]{y - 2};
\end{displaymath}
\begin{tikzpicture}
	\begin{axis}[axis lines=middle,grid=both,no markers,xlabel=$x$,ylabel=$y$,
		xmin=-4,xmax=6,
		ymin=-4,ymax=6,
		xtick={-10,...,10},
		ytick={-10,...,10},
        ]

		\draw[black] (-10, -10) -- (10, 10);
		\addplot+[thick, domain=-5:5, samples=100] {x^3 + 2};		
		% hack
		\addplot+[thick, domain=-10:2, samples=100] {-(2 - x)^(1/3)};
		\addplot+[thick, red, domain=2:10, samples=100] {(x - 2)^(1/3)};
    \end{axis}
\end{tikzpicture}
\begin{displaymath}
Y = (-\infty, \infty); X = (-\infty, \infty);
\end{displaymath}

\ex{22.}
\begin{displaymath}
y = 2^{3x - 4}
\end{displaymath}
\begin{displaymath}
3x - 4 = \log_2 y
\end{displaymath}
\begin{displaymath}
x = \frac{\log_2 y + 4}{3}; y \in (0; \infty)
\end{displaymath}
\begin{tikzpicture}
	\begin{axis}[axis lines=middle,grid=both,no markers,xlabel=$x$,ylabel=$y$,
		xmin=-2,xmax=4,
		ymin=-2,ymax=4,
		xtick={-5,...,5},
		ytick={-5,...,5},
        ]

		\draw[black] (-10, -10) -- (10, 10);
		\addplot+[thick, domain=-2:4, samples=100] {2^(3*x - 4)};
		\addplot+[thick, domain=0:1.5, samples=1000] {(log2(x) + 4)/3};
		\addplot+[thick, red, domain=1.2:5, samples=100] {(log2(x) + 4)/3};
    \end{axis}
\end{tikzpicture}
\begin{displaymath}
Y = (0; \infty); X = (-\infty; \infty)
\end{displaymath}

\ex{23.}
\begin{displaymath}
y = 8\pi + 8arc\cot \frac{3x - 1}{2}; (y \in (8\pi; 16\pi))
\end{displaymath}
\begin{displaymath}
arc\cot \frac{3x - 1}{2} = \frac{y}{8} - \pi
\end{displaymath}
\begin{displaymath}
\frac{3x - 1}{2} = \cot(\frac{y}{8} - \pi)
\end{displaymath}
\begin{displaymath}
3x = 2\cot(\frac{y}{8} - \pi) + 1
\end{displaymath}
\begin{displaymath}
x = \frac{1}{3} + \frac{2}{3} \cot \frac{y}{8}; y \in (0; 8\pi)
\end{displaymath}
\begin{tikzpicture}
	\begin{axis}[axis lines=middle,grid=both,no markers,xlabel=$x$,ylabel=$y$,
		xmin=-20,xmax=60,
		ymin=-20,ymax=60,
		xtick={-20,-10,...,50,60},
		ytick={-20,-10,...,50,60},
        ]

		\draw[black] (-30, -30) -- (70, 70);
		\addplot+[thick, domain=-30:70, samples=100] {8*pi + 8 * rad(90-atan((3*x-1)/2))};
		\addplot+[thick, domain=8*pi+0.001:16*pi, samples=1000] {1/3 + 2/3 * cot(deg(x/8))};
    \end{axis}
\end{tikzpicture}
\begin{displaymath}
Y = (8\pi; 16\pi); X = (-\infty, \infty)
\end{displaymath}

\ex{24.}
\begin{displaymath}
y = 1 + \log |x - 2|; x \in (-\infty; 2)
\end{displaymath}
\begin{displaymath}
y = 1 + \log (2 - x)
\end{displaymath}
\begin{displaymath}
\log (2 - x) = y - 1
\end{displaymath}
\begin{displaymath}
2 - x = 10^{y - 1}
\end{displaymath}
\begin{displaymath}
x = 2 - 10^{y - 1}
\end{displaymath}
\begin{tikzpicture}
	\begin{axis}[axis lines=middle,grid=both,no markers,xlabel=$x$,ylabel=$y$,
		xmin=-4,xmax=4,
		ymin=-4,ymax=4,
		xtick={-10,...,10},
		ytick={-10,...,10},
        ]

		\draw[black] (-10, -10) -- (10, 10);
		\addplot+[thick, domain=-5:1.7, samples=100] {1 + log10(2 - x)};	
		\addplot+[thick, blue, domain=1.5:2, samples=800] {1 + log10(2 - x)};
		\addplot+[thick, red, domain=-4:3, samples=100] {2 - 10^(x - 1)};
    \end{axis}
\end{tikzpicture}
\begin{displaymath}
Y = (-\infty; \infty); X = (-\infty; 2)
\end{displaymath}

\ex{25.}
\begin{displaymath}
y = 1 + \arccos(1 - x); (x \in [0, 2])
\end{displaymath}
\begin{displaymath}
\arccos(1 - x) = y - 1
\end{displaymath}
\begin{displaymath}
1 - x = \cos(y - 1)
\end{displaymath}
\begin{displaymath}
x = 1 - \cos(y - 1); y \in [1; \pi + 1]
\end{displaymath}
\begin{tikzpicture}
	\begin{axis}[axis lines=middle,grid=both,no markers,xlabel=$x$,ylabel=$y$,
		xmin=-1,xmax=5,
		ymin=-1,ymax=5,
		xtick={-10,...,10},
		ytick={-10,...,10},
        ]

		\draw[black] (-10, -10) -- (10, 10);
		\addplot+[thick, domain=0:2, samples=100] {1 + rad(acos(1 - x))};
		\addplot+[thick, domain=1:1+pi, samples=800] {1 - cos(deg(x - 1))};
    \end{axis}
\end{tikzpicture}
\begin{displaymath}
Y = [1; 1 + \pi]; X = [0, 2];
\end{displaymath}

\ex{26.}
\begin{displaymath}
y = \frac{1}{2} \arcsin \frac{x}{3}; (x \in [-3; 3])
\end{displaymath}
\begin{displaymath}
\arcsin \frac{x}{3} = 2y;
\end{displaymath}
\begin{displaymath}
\frac{x}{3} = \sin 2y;
\end{displaymath}
\begin{displaymath}
x = 3\sin 2y; 
\end{displaymath}
\begin{tikzpicture}
	\begin{axis}[axis lines=middle,grid=both,no markers,xlabel=$x$,ylabel=$y$,
		xmin=-3.5,xmax=3.5,
		ymin=-3.5,ymax=3.5,
		xtick={-10,...,10},
		ytick={-10,...,10},
        ]

		\draw[black] (-10, -10) -- (10, 10);
		\addplot+[thick, domain=-3:3, samples=100] {1/2 * rad(asin(x / 3))};
		\addplot+[thick, domain=-pi/4:pi/4, samples=100] {3*sin(deg(2*x))};
    \end{axis}
\end{tikzpicture}
\begin{displaymath}
Y = [-\frac{\pi}{4}; \frac{\pi}{4}]; X = [-3; 3]
\end{displaymath}

\end{document}